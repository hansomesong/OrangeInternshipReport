% Chapter Template

\chapter{INTERNSHIP’S OBJECTIVE} % Main chapter title

\label{Chapter3} % Change X to a consecutive number; for referencing this chapter elsewhere, use \ref{ChapterX}

\lhead{Chapter 3. \emph{INTERNSHIP’S OBJECTIVE}} % Change X to a consecutive number; this is for the header on each page - perhaps a shortened title

As mentioned before, in terms of bridging semantic gap for VMI, the most popular method is delivery pattern. In this field, there 
exist plenty of VMI applications for various purposes. Although this method could help to effectively mitigate the semantic gap, it poses other 
problems such as portability, non-robust against circumstances technologies, etc. For example, if we use LibVMI to help parse guest Linux kernel
data structure, the system symbol file is required firstly to be transferred to host machine. In case of kernel’s update in guest, this file 
transfer needs to be executed again. In this context, we want to explore the potential of derivative pattern, which relies on guest hardware 
architecture to extract useful information. Pfoh has declared in his work \cite{Reference7} that:

\textcolor{blue}{
  “These methods allow one to enumerate the running processes, monitor system calls, or track network connections 
  on a per-process basis in a completely guest OS agnostic manner.” 
}

His declaration is rather interesting and attractive even though Pfoh has not given more explanation to argue his idea. Thus, the main 
objective of this internship is to prove this idea and develop a prototype implementation. The prototype should present the following features:

\begin{itemize}
    \item Capable of tracking network connection on a per-process
    \item Works in derivative method and independent of guest OS
 \end{itemize}

In conclusion, it’s a “netstat-like” utility by leveraging derivative pattern to bridge semantic gap and works out of monitored guest.

We plan to achieve our determined goal by the following steps:

\begin{itemize}
    \item Install and manipulate Nitro to see how derivative pattern works
    \item Study which component in KVM virtualization platform is in charge of virtual networking
    \item Study how to enumerate running process in a guest agnostic manner
    \item Study how to correlate each network connection with identified running process 
 \end{itemize}
