% Chapter Template

\chapter{INTRODUCTION} % Main chapter title

\label{Chapter1} % Change X to a consecutive number; for referencing this chapter elsewhere, use \ref{Chapter1}

\lhead{Chapter 1. \emph{INTRODUCTION}} % Change X to a consecutive number; this is for the header on each page - perhaps a shortened title


Virtual machine Introspection, also known as VMI, is an emerging technology largely used in virtual data center in recent years, 
for the purpose of building a wide range of agentless VM monitored applications such as intrusion detection system \cite{Reference1}, 
virtual firewall \cite{Reference2}, etc. According to Pfoh \cite{Reference5}, there exist three patterns to help build VMI applications
: in-band pattern, out-of-pattern and derivative pattern. In-band pattern needs an agent installed in monitored guest, thus it is not our 
investigation focus. Out-of-band is the most popular VMI approach but with some problems of portability.    Derivative pattern, relying on 
hardware architecture information (MMU state, control registers in vCPU, etc.) to infer guest running state, presents a better portability 
feature and there is no much work in this field. Based on this classification, I suggest adding a new one: reutilization pattern, because 
recent advance shows that the semantic gap could be largely narrowed by reusing the exercised code from a trusted OS kernel. We plan to make 
a full state-of-the-art for currently existing VMI technologies. This article is aimed to record all the exploration process in this road.
